\chapter{Evaluation des Konzepts anhand des Prototyps}
\label{sec:eva}
Dieses Kapitel beschäftigt sich mit der Bewertung des Konzepts durch Probanden. Diese haben dabei den Prototypen im Rahmen einer Pilotstudie bewertet. Außerdem soll das beobachtete Verhalten der Probanden analysiert werden, um daraus Schlüsse auf mögliche Verbesserungen ziehen zu können.


\section{Studienablauf}
\label{sec:eva:aufgaben}

Der Ablauf der Studie kann in verschiedene Abschnitte unterteilt werden. Eingangs wurde den Teilnehmern kurz erklärt, wieso diese Studie durchgeführt wird und was das Ziel der Arbeit ist. Außerdem wurde darauf hingewiesen, dass einige Funktionen nur prototypisch umgesetzt wurden und die Teilnehmer sich somit vorstellen sollte, dass der Roboter die entsprechende Aktion korrekt ausgeführt hat, insofern diese auch erkannt wurde. Um einen Einstieg zu ermöglichen, wurde den Teilnehmern mitgeteilt, auf welches Signalwort das System reagiert. Weitere Befehle sowie deren Funktionen wurden nicht erläutert, damit auch untersucht werden kann, wie gut die Teilnehmer den Roboter ohne weitere Vorkenntnisse steuern können. Außerdem wurden alle Probanden dazu aufgefordert, während der Durchführung der Aufgaben laut zu denken. Diese geäußerten Gedanken wurden später wieder in dem kurzen persönlichen Gespräch verwendet, um tiefere Eindrücke zu erlangen. Die Durchführung der Aufgaben sowie die Beantwortung der Fragen wurde dabei auf Video aufgezeichnet. \\
Während der Durchführung der Aufgaben befand sich die gesamte Zeit er Studienleiter in Raum, um auch für die spätere Befragung einen eigenen Eindruck zu erlangen. Außerdem konnte so bei Bedarf helfend eingegriffen werden. Die quantitativen Fragen wurden anschließend auf einem Tablet mithilfe von Onlineformularen ausgeführt. Die anschließende qualitative Auswertung fand in Form eines Gesprächs statt.

\subsection{Untersuchungsziel}

Die aufgestellte Hypothese für die Studie ist: \glqq Die Nutzung des Systems ist für den Nutzer selbsterklärend und fühlt sich natürlich an. \grqq{}
Dies ist besonders wichtig, damit die in Kapitel \ref{sec:einsatzszenarien} erläuterten Aufgaben optimal erfüllt werden können. \\
Außerdem wurden soll auch noch folgende Subhypothese betrachtet werden: \glqq Ein stärkerer Fokus auf den Datenschutz überzeugt auch Nutzer, die bislang keinen Sprachassistenten verwenden.\grqq{} \\

\subsection{Aufgabenstellung}
Jeder Proband hat dieselben Aufgaben erhalten, auf Basis derer anschließend die Bewertung vorgenommen werden sollte. Dafür wurden die Aufgaben so allgemein wie möglich formuliert, damit keine suggestiven Handlungen entnommen werden können, die am Ende das Stdueinergebnis verfälschen. \\
Folgende Aufgaben wurden den Teilnehmern vorgelegt: 
\begin{my_list_num}
  \item Sorgen Sie dafür, dass sich der Roboter nach links beziehungsweise rechts dreht.
  \item Lassen Sie den Roboter in die Küche fahren.
  \item Teilen Sie dem Roboter mit, dass er wieder zu Ihnen zurückkehren soll.
  \item Beauftragen Sie den Roboter damit, Ihnen Kaffee zu holen.
  \item Teilen Sie dem Roboter mit, dass Sie für Ihren Kaffee noch Milch benötigen. Während der Roboter die Aufgabe ausführt, ändert sich Ihre Meinung und sie möchten doch keine Milch. Sorgen Sie dafür, dass der Roboter die aktuelle Handlung abbricht.
  \item Gehen Sie davon aus, dass Ihnen der Roboter im Weg steht. Teilen Sie ihm nun also mit, dass er Ihnen Platz machen soll.
\end{my_list_num}

\subsection{Quantitative Fragen}

Die quantitative Befragung der Nutzer hat auf Basis der System Usability Scale von Brooks et al. \cite{brooke1996sus} stattgefunden. Diese Entscheidung wurde getroffen, da diese Fragen nicht zu umfangreich sind, es aber gleichzeitig erlauben, die Ergebnisse gut einzuordnen. Dies ist besonders dadurch der Fall, dass diese Skala schon häufig zum Einsatz gekommen ist und es somit entsprechenden Einordnungen der erzielten Punktzahl und der Güte des Produkts gibt. \\
Dabei wird der Nutzer mit zehn verschiedenen Aussagen konfrontiert, die er auf eine Skala von eins (stark Zustimmung) bis 5 (starke Ablehnung) bewerten soll. Für die Berechnung eines Gesamtergebnisses werden diese Punkte addiert. Da jedoch die Aussagen alternieren, ist manchmal hohe Zustimmung besser, während in anderen Fällen Ablehnung besser ist. Dafür wird bei jeder geradzahligen Frage der erhaltene Wert von fünf subtrahiert, bei ungeraden Fragen der Wert entsprechend der Skala entnommen und um eins verringert. Um danach eine Einordnung vornehmen zu können, wird dieser Wert mit dem Faktor 2,5 multipliziert. Wenn das Endergebnis dabei über 80,3 Punkten liegt, kann dies als sehr gutes Ergebnis betrachtet werden. Die folgenden zehn Aussagen werden dabei durch die Nutzer bewertet:
\begin{my_list_num}
  \item I think that I would like to use this system frequently.
  \item  I found the system unnecessarily complex.
  \item I thought the system was easy to use.
  \item I think that I would need the support of a technical person to be able to use this system.
  \item I found the various functions in this system were well integrated.
  \item I thought there was too much inconsistency in this system.
  \item I would imagine that most people would learn to use this system very quickly.
  \item I found the system very cumbersome to use.
  \item I felt very confident using the system.
  \item I needed to learn a lot of things before I could get going with this system.
\end{my_list_num}

Außerdem wurden diese Fragen noch um die Folgenden erweitert, damit systemspezifischere Aspekte genauer betrachtet werden können:
\begin{my_list_num}
  \item The spoken answers by the system were clear and natural.
  \item The system responded quickly to my requests.
  \item The interaction with the system felt natural.
  \item I had no problems using the wakedword.
  \item I found the delay until the system answered disturbing in the interaction.
  \item I am already using a voice assistant.
  \item Privacy is very important for me.
  \item I would us a voice assistant more often, if my privacy is guranteed.
\end{my_list_num}

Die Fragen wurden dabei absichtlich auf Englisch gehalten, da die Interaktion mit dem Roboter ebenfalls auf Englisch stattfindet und somit Gedankengänge möglichst konsistent erhalten bleiben. Zusätzlich dazu wurden die Teilnehmer noch um einige persönliche Angaben gebeten, um einen besseren Eindruck der Durchmischung zu erhalten. \\
Dazu wurden neben dem Geschlecht auch Alter, höchste Qualifikation, aktuelle Arbeitssituation sowie eine Selbsteinschätzung über die Technikversiertheit erfragt.
\subsection{Qualitative Fragen}

Die qualitativen Fragen konnten nicht genauso eindeutig wie die quantitativen formuliert werden. Dies hat den Grund, dass sie sich nicht ausschließlich an den Hypothesen orientieren. Vielmehr sollen sie helfen, die Gedanken der Probanden über das System besser zu verstehen. Diesen Prozess unterstützen auch die zuvor laut gedachten Überlegungen der Teilnehmer. So konnte spezifisch auf Besonderheiten bei jedem einzelnen Teilnehmer eingegangen werden. Allerdings gab es auch hier einen gewissen Fokus. \\
So war es von großen Interesse, herauszufinden, aus welchen Gründen Sprachassistenten genutzt beziehungsweise ignoriert werden. Auch wurde nochmals vertieft hinterfragt, inwiefern der Datenschutz auf das Nutzungsverhalten Einfluss nehmen kann. Dazu wurde auch bei Bedarf die Funktionsweise des getesteten System genauer beschrieben und mögliche Nachteile eines solchen Systems gegenüber Systemen, die die Privatssphäre weniger stark respektieren. Auch sollte herausgefunden werden, warum sich die Interaktion für die einzelnen Probanden so angefühlt hat, wie dies der Fall war.

\section{Auswertung der Antworten}
\label{sec:eva:antworten}

\subsection{Teilnehmer}

Die Teilnehmer wurden aus dem persönlichen Umfeld des Autors gewählt und stellen somit keine repräsentatives Teilnehmerfeld dar. Da es sich bei der vorliegendenen Studie allerdings um eine Pilotstudie handelt, sollte dies aktzeptabel sein. Es wurde bestmöglich darauf geachtet, dass die Teilnehmer heterogen sind. Dabei hat sich folgenden Struktur ergeben: \\
Alter \\
geschlecht \\
Ausbildung \\
etc

\section{Analyse des Verhaltens der Probanden}
\label{sec:eva:verhalten}

\section{Zusammenfassung}
\label{sec:eva:zusammenfassung}